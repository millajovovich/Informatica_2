\documentclass{letter}
\usepackage[utf8]{inputenc}
\usepackage[spanish]{babel}
\usepackage{graphicx}
\usepackage{geometry} 
\geometry{
	top=2.5cm, 
	bottom=3cm, 
	left=3cm, 
	right=3cm, 
}
\renewcommand{\baselinestretch}{1.3}
\begin{document}
\includegraphics[width=0.5\textwidth]{logo-udea.png}\\

\opening{Sebastian Balbin Rivera.}
“Cuando las computadoras igualen la capacidad de cálculo\\
del cerebro humano, necesariamente lo superarán.” \\
Raymond Kurzweil\\\\
A finales del siglo XVII el filósofo y matemático alemán Gottfried Wilhelm Leibniz presento el modelo definitivo de su máquina la stepped reckoner, que realizaba las cuatro operaciones aritméticas; suma, resta, división y multiplicación, siendo clave para la implementación de otras máquinas usando la rueda de Leibniz. Tiempo después en 1703 Leibniz daría un paso de suma importancia presentando el sistema binario que dependía de dos estados; unos y ceros, proponiéndolo como una manera para realizar operaciones de una manera más fácil, pero no fue de mucha atención ya que el sistema decimal tendía a ser más eficiente y fácil. Después de esto alrededor de 1837 Charles Babbage diseño la maquina analítica que era una maquina demasiado innovadora para aquella época, talvez un siglo adelantada a su tiempo; esta recibiría tarjetas perforadas  (que vendrían a ser como la programación de una computadora actual para esa época) ya era un mecanismo implementado en otro tipo de maquinarias no orientadas a la computación en esos días, lastimosamente no se pudo crear por falta de financiamiento y por la falta de tecnología para construirla; sin embargo estos fueron los primeros pasos de un método de computación. Fue a mediados del siglo XIX que George Boole lógico y matemático ingles representando premisas con unos y ceros, intento trabajarlas algebraicamente para resolver ejercicios de lógica proposicional y llegar a una conclusión siendo esto uno de los pasos claves a futuro. \\\\ 
Entrando entonces a una de las etapas más importantes de lo que caracterizo la computación llegamos a la crisis de los fundamentos, una etapa en la que las matemáticas se estaban estancando a finales del XIX y principios del XX, siendo entonces Georg Cantor uno de los pioneros en esta crisis con sus planteamientos en la teoría de conjuntos y sus cardinalidades que luego sería una de las razones de muchos debates entre la comunidad científica; la comunidad aceptaba como muy sólida la teoría presentada por Cantor hasta que analizando ha profundidad se dieron cuenta de que se podían generar contradicciones y paradojas, lo que genero un impacto increíble en toda la comunidad, pues estaba entrando a jugar un papel muy importante la infinidad haciéndose presente en el trabajo de Cantor que luego sería solucionado pero incluyendo nuevas contradicciones y paradojas.\\\\
Unos de los siguientes pasos más importantes en esta crisis fue dado por el matemático David Hilbert con su Programa de Hilbert, el cual establecía que un sistema axiomático no debía tener contradicciones, ser finito y completo, siendo esta una de las causas que termino separando en facciones a la comunidad científica en el logicismo, intuicionismo y el formalismo, debatiendo por años, hasta que en 1931 Kurl Godel demostró que no es posible un sistema axiomático por los siguientes motivos  (afirmados por Alejandro Ortiz en su artículo, “crisis en los fundamentos de la matemática”):\\
- si la teoría axiomática de conjuntos es consistente, entonces existen teoremas que no pueden ser probados ni refutados.\\
- no existe ningún procedimiento constructivo que pruebe que la teoría axiomática de conjuntos sea consistente.\\
marcando esto uno de los puntos más importantes en las matemáticas.\\\\
Terminada la crisis, que ya veremos la implicación que tuvo sobre la computación, y los mecanismos que se venían desarrollando lentamente sobre computación, llegamos a Claude Shannon un ingeniero eléctrico que unió la lógica con la computación para descubrir que por medio de interruptores podía representar las preposiciones lógicas y no solo esto, si no resolverlas por medio de operaciones aritméticas usando el álgebra creada por George Boole; esto fue lo más fundamental ya que determinaría desde ese momento la computación moderna que no es mas que algebra bool y lógica. Unido a esto con los aires de la segunda guerra mundial llego otro magnifico Matemático llamado Alan Turing, haciendo historia no solo con su máquina enigma en la segunda guerra mundial sino que también con su aporte a las matemáticas  y a la computación misma, resolviendo  junto con Alonzo Church el Entscheidungsproblem (problema de decisión) que fue  planteado por David Hilbert y Wilhelm Ackermann en 1928, demostrando que si no es posible computar el algoritmo no se puede afirmar si tiene solución, siendo todo esto lo que marco el nacimiento de la computación moderna como la conocemos hoy día desde problemas matemáticos que no tenían enfoque computacional hasta la unión de ambos en uno.




\\
%\closing{asi termina}
\end{document}
